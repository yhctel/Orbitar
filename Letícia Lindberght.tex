\documentclass[
	% -- opções da classe memoir --
	12pt,				% tamanho da fonte
	openright,			% capítulos começam em pág ímpar (insere página vazia caso preciso)
	oneside,			% para impressão em frente e verso. Oposto a oneside
	a4paper,			% tamanho do papel.
	% -- opções da classe abntex2 --
	% As opções chapter=TITLE, section=TITLE etc. foram removidas para corrigir erros de compilação
	% com o pacote hyperref (Token not allowed in a PDF string). O abntex2 já formata o sumário
	% em maiúsculas por padrão.
	% -- opções do pacote babel --
	english,			% idioma adicional para hifenização
	brazil				% o último idioma é o principal do documento
	]{abntex2}


% ---r
% Pacotes básicos
% ---
%\usepackage{lmodern}			% Usa a fonte Latin Modern
\usepackage{mathptmx}			% Usa a fonte Times New Roman
%\usepackage{helvet}			% Fonte Parecida com Arial
\usepackage[T1]{fontenc}		% Selecao de codigos de fonte.
\usepackage[utf8]{inputenc}		% Codificacao do documento (conversão automática dos acentos)
\usepackage{lastpage}			% Usado pela Ficha catalográfica
\usepackage{indentfirst}			% Indenta o primeiro parágrafo de cada seção.
\usepackage{graphicx}			% Inclusão de gráficos
\usepackage{subcaption}			% Inclusão de gráficos lado a lado
\usepackage{microtype} 			% para melhorias de justificação
\usepackage{tabularx,ragged2e}	% Para inserir tabelas
\usepackage{multirow}			% Para mesclar células
\usepackage[dvipsnames,table,xcdraw]{xcolor}		% Permite adicionar cores nas linhas de tabelas. Carregado no lugar do pacote 'color' por ser mais completo.
\usepackage{fancyvrb}			% Permite adicionar arquivos de texto
\usepackage[portuguese, ruled, linesnumbered]{algorithm2e} % Uso de algoritmos
\usepackage{amsfonts}			% Permite usar notação de conjuntos
\usepackage{amsmath}			% Permite citar equações
\usepackage{amsthm}				% Permite criar teoremas e experimentos
\usepackage[font={bf, small}, labelsep=endash, labelfont=bf]{caption}	% Faz legenda de figuras ficarem em negrito
\usepackage{cancel}				% Permite fazer expressão tendendo a zero
\usepackage{epstopdf}			% Converte eps para pdf
\usepackage[final]{pdfpages}
\usepackage{hyphenat}
% \usepackage{fancyhdr} % -- COMENTADO: O pacote fancyhdr não é recomendado com a classe memoir, que possui suas próprias ferramentas de customização.
\usepackage{enumitem}

\newcolumntype{L}{>{\RaggedRight\arraybackslash}X}
% ---
		
% ---
% Pacotes adicionais, usados apenas no âmbito do Modelo Canônico do abnteX2
% ---
\usepackage{lipsum}				% para geração de dummy text
% ---

% ---
% Pacotes de citações
% ---
%\usepackage[brazilian,hyperpageref]{backref}	 % Paginas com as citações na bibl
\usepackage[alf, abnt-emphasize=bf]{abntex2cite}	% Citações padrão ABNT

% ---
% Customizações para o layout da UFPA
% ---
\usepackage{modelo-ufpa/ufpa}

% --- Bloco de comandos para justificação removido ---
% Os comandos \tolerance, \hyphenpenalty, \sloppy etc. foram removidos.
% O pacote 'microtype' já está carregado e oferece uma solução tipograficamente
% superior para problemas de justificação (hifenação e espaçamento).
% ---

\renewcommand{\ABNTEXchapterfontsize}{\normalsize}
\renewcommand{\ABNTEXsectionfontsize}{\normalsize}
\renewcommand{\ABNTEXsubsectionfontsize}{\normalsize}
\renewcommand{\ABNTEXsectionfont}{}
\renewcommand{\ABNTEXsubsectionfont}{}

\renewcommand{\chaptername}{ }
%\renewcommand{\familydefault}{\sfdefault} % usar apenas se usar a fonte helvet

% Muda o título de lista de ilustrações para lista de figuras
\addto\captionsbrazil{%
  \renewcommand{\listfigurename}%
    {Lista de Ilustrações}%
	\renewcommand{\listtablename}%
    {Lista de Tabelas}%
}

% --- Estilo de página customizado usando as ferramentas da classe 'memoir' ---
% Substituindo a implementação que usava o pacote 'fancyhdr'
\makepagestyle{ultima}
\makeoddhead{ultima}{}{}{\thepage}
\makeevenhead{ultima}{}{}{\thepage} % Garante que funcione em oneside e twoside
\makeheadrule{ultima}{0pt}{0pt} % Remove a linha do cabeçalho
\makefootrule{ultima}{0pt}{0pt} % Remove a linha do rodapé
% ---

% Permite utilizar figuras sem precisar colocar o caminho absoluto
\graphicspath{{imagens/}}

% Define o ambiente de experimentos
\theoremstyle{definition}
\newtheorem{experimento}{Experimento}[section]
\newcommand{\experimentoautorefname}{Experimento}

% --- 
% CONFIGURAÇÕES DE PACOTES
% --- 

% ---
% Configurações do pacote backref
% Usado sem a opção hyperpageref de backref
%\renewcommand{\backrefpagesname}{Citado na(s) página(s):~}
% Texto padrão antes do número das páginas
%\renewcommand{\backref}{}
% Define os textos da citação
%\renewcommand*{\backrefalt}[4]{
%	\ifcase #1 %
%		Nenhuma citação no texto.%
%	\or
%		Citado na página #2.%
%	\else
%		Citado #1 vezes nas páginas #2.%
%	\fi}%
% ---

% --- Informações da capa, folha de rosto e ficha catalográfica ---
\universidade{Centro Universitário Serra dos Órgãos \- UNIFESO}
\instituto{Direção Acadêmica de Ciências Humanas e Tecnológicas \- DACHT}
\curso{Curso de Bacharelado em Ciência da Computação}
\titulo{Orbitar: Uma plataforma web para o reaproveitamento de eletrônicos e promoção da economia circular}
\autor{Letícia Lindberght da Costa}
\local{Teresópolis}
\data{2025}
\orientador{Prof\. Nome do Orientador}
\tipotrabalho{Monografia}
\preambulo{Trabalho de Conclusão de Curso apresentado ao Centro Universitário Serra dos Órgãos como requisito obrigatório para obtenção do título de Bacharel em Ciência da Computação.}
\sobrenome{Lindberght}
\nome{Letícia}
\palavraschave{Palavra-chave1, Palavra-chave2, Palavra-chave3}

\datadadefesa{Data da Defesa: 28 de Novembro de 2025}
%\conceito{Conceito: Excelente}
\faculdadedoorientador{Faculdade do Orientador}
\titulacaodoorientador{MSc}
\faculdadedocoorientador{Faculdade do Coorientador}
\titulacaodocoorientador{DSc}
\primeiromembrodabanca{Nome do Primeiro Membro da Banca}
\titulacaodoprimeiromembro{DSc}
\faculdadedoprimeiromembrodabanca{Faculdade do Primeiro Membro da Banca}
\segundomembrodabanca{Nome do Segundo Membro da Banca}
\titulacaodosegundomembro{DSc}
\faculdadedosegundomembrodabanca{Faculdade do Segundo Membro da Banca}
% ---

% ---
% Configurações de aparência do PDF final

% alterando o aspecto da cor azul
\definecolor{blue}{RGB}{41,5,195}

% informações do PDF
\hypersetup{
    pdftitle={\texorpdfstring{\imprimirtitulo}{Orbitar \- Plataforma Web de Reaproveitamento de Eletrônicos}}, 
    pdfauthor={\texorpdfstring{\imprimirautor}{Letícia Lindberght da Costa}},
    pdfsubject={\texorpdfstring{\imprimirpreambulo}{Trabalho de Conclusão de Curso}},
    pdfkeywords={\texorpdfstring{\imprimirpalavraschave}{Palavra-chave1, Palavra-chave2, Palavra-chave3}},
    colorlinks=true,
    linkcolor=black,
    citecolor=black,
    filecolor=magenta,
    urlcolor=black,
    bookmarksdepth=4,
    breaklinks=true
}
% --- 

% --- 
% Espaçamentos entre linhas e parágrafos 
% --- 

% O tamanho do parágrafo é dado por:
\setlength{\parindent}{1.5cm}

% Controle do espaçamento entre um parágrafo e outro:
\setlength{\parskip}{0.2cm} % tente também \onelineskip

% ---
% compila o indice
% ---
\makeindex
% ---

% ----
% Início do documento
% ----
\begin{document}

% AVISO: O comando \nocite abaixo irá gerar um erro de "Citation undefined" se a chave 'Vrs:2025'
% não existir no seu arquivo bibliografia.bib. Certifique-se de que ela está lá.
\nocite{Vrs:2025}
% Seleciona o idioma do documento (conforme pacotes do babel)
%\selectlanguage{english}
\selectlanguage{brazil}

% Retira espaço extra obsoleto entre as frases.
\frenchspacing 


% ----------------------------------------------------------
% ELEMENTOS PRÉ-TEXTUAIS
% ----------------------------------------------------------
% \pretextual

% ---
% Capa
% ---
\imprimircapa{}
% ---

% ---
% Folha de rosto
% ---
\imprimirfolhaderosto{}
% ---

% ---
% Inserir a ficha bibliografica
% ---
% A biblioteca da universidade lhe fornecerá um PDF
% com a ficha catalográfica definitiva após a defesa do trabalho. Quando estiver
% com o documento, salve-o como PDF no diretório do seu projeto e substitua todo
% o conteúdo de implementação deste arquivo pelo comando abaixo:

%\begin{fichacatalografica}
%    \includepdf{fichacatalografica.pdf}
%\end{fichacatalografica}

\newpage
% ---
% ---

% ---
% Inserir folha de aprovação
% ---
%
\begin{folhadeaprovacao}
\imprimirfolhadeaprovacao{}
\end{folhadeaprovacao}
% ---

% ---
% Dedicatória
% ---

% ESCREVA A SUA DEDICATORIA A DEDICATORIA QUE SE ENCONTRA NO ARQUIVO E APENAS UM EXEMPLO. ESCOLHA A DEDICATORIA QUE MAIS LHE AGRADAR. LEMBRE-SE DE UTILIZAR AS '\\' PARA PULAR LINHAS

\begin{dedicatoria}
    \vspace*{\fill}
    \flushright{}
    %\noindent
    \textit{Este trabalho é dedicado às crianças adultas que,
	quando pequenas, sonharam em se tornar cientistas.
	\- Lauro César em abnTeX2}
\end{dedicatoria}
% ---

% ---
% Agradecimentos
% ---
\begin{agradecimentos}

Escreva aqui o seus agradecimentos da maneira que melhor lhe convêm.

\end{agradecimentos}
% ---

% ---
% Epígrafe
% ---

% ESCREVA A SUA EPÍGRAFE A MESMA QUE SE ENCONTRA NO ARQUIVO E APENAS UM EXEMPLO. ESCOLHA A EPÍGRAFE QUE MAIS LHE AGRADAR. LEMBRE-SE DE UTILIZAR AS '\\' PARA PULAR LINHAS

\begin{epigrafe}
    \vspace*{\fill}
	\begin{flushright}
		\textit{``Que todos os nossos esforços estejam sempre focados no desafio à impossibilidade.\\ Todas as grandes conquistas humanas vieram daquilo que parecia impossível\.''\\
		(Charles Chaplin)}
	\end{flushright}
\end{epigrafe}
% ---

% ---
% RESUMOS
% ---

% ---
% inserir lista de ilustrações
% ---
% UTILIZE CASO HAJA FIGURAS NA MONOGRAFIAS. EM QUASO DE AUSENCIA DE FIGURAS COMENTAR COM AS 3 LINHAS DE CODIGOS UTILIZANDO O '%'.
\pdfbookmark[0]{\listfigurename}{lof}
\listoffigures*
\cleardoublepage{}
% ---

% ---
% inserir lista de quadros
% ---
% UTILIZE CASO HAJA QUADROS NA MONOGRAFIAS. EM QUASO DE AUSENCIA DE QUADROS COMENTAR COM AS 3 LINHAS DE CODIGOS UTILIZANDO O '%'.
\pdfbookmark[0]{\listofquadrosname}{loq}
\listofquadros*
\cleardoublepage{}
% ---

% ---
% inserir lista de tabelas
% ---
% UTILIZE QUASE HAJA TABELAS NA MONOGRAFIAS. EM QUASO DE AUSENCIA DE TABELAS COMENTAR COM AS 3 LINHAS DE CODIGOS UTILIZANDO O '%'.
\pdfbookmark[0]{\listtablename}{lot}
\listoftables*
\cleardoublepage{}
% ---

% ---
% inserir lista de algoritmos
% ---
% UTILIZE CASO HAJA ALGORITMOS NA MONOGRAFIAS. EM QUASO DE AUSENCIA DE ALGORITMOS COMENTAR COM AS 3 LINHAS DE CODIGOS UTILIZANDO O '%'.
\pdfbookmark[0]{\listalgorithmcfname}{loa}
\imprimirlistadealgoritmos{}
\cleardoublepage{}
% ---

% ---
% inserir lista de abreviaturas e siglas
% ---
% DEVE SER PREENCHIDA A MÃO LEMBRE-SE DE MANTER EM ORDEM ALFABETICA.
\begin{siglas}
  \item[ABNT] Associação Brasileira de Normas Técnicas
\end{siglas}
% ---

% ---
% inserir lista de símbolos
% ---
% DEVE SER PREENCHIDA A MÃO LEMBRE-SE DE MANTER EM ORDEM ALFABETICA.
%\begin{simbolos}
%\item[\(\Theta\)] Letra grega maiúscula theta
%\end{simbolos}

% ---

% resumo em português
\setlength{\absparsep}{18pt} % ajusta o espaçamento dos parágrafos do resumo
\begin{resumo}
Escreva aqui o resumo do seu trabalho. Lembre-se escreva o resumo por ultimo na monografia. 

\textbf{Palavras-chave}: \imprimirpalavraschave{}
\end{resumo}

% resumo em inglês
\begin{resumo}[Abstract]
 \begin{otherlanguage*}{english}
   Escreva aqui o seu abstract (Resumo em inglês)

   \vspace{\onelineskip}

   \noindent 
   \textbf{Keywords}: Keywords1. Keywords2. Keywords3.
 \end{otherlanguage*}
\end{resumo}

% ---
% inserir o sumario
% ---
\pdfbookmark[0]{\contentsname}{toc}
\tableofcontents*
\cleardoublepage{}
% ---

% ----------------------------------------------------------
% ELEMENTOS TEXTUAIS
% ----------------------------------------------------------
\textual{}
\pagestyle{simple}

% ----------------------------------------------------------
% Introdução
% ----------------------------------------------------------
\chapter[INTRODUÇÃO]{INTRODUÇÃO}

As Tecnologias da Informação e Comunicação \(TIC\) são pilares da sociedade contemporânea, catalisando o progresso social e econômico. Contudo, o modelo de produção que as sustenta, predominantemente linear e focado no consumo cíclico, gerou um passivo ambiental de proporções críticas: o lixo eletrônico, ou e-waste. Este fenômeno é o resultado direto de um sistema caracterizado pela obsolescência programada e por padrões de consumo insustentáveis, onde equipamentos são descartados em um ritmo alarmante, muitas vezes ainda em plenas condições de uso.

A magnitude deste problema é documentada por múltiplas fontes globais. O relatório \textit{Global E-waste Monitor 2024} revela que a geração de lixo eletrônico atingiu um recorde histórico, crescendo a uma taxa cinco vezes superior à da reciclagem documentada, com dezenas de milhões de toneladas descartadas anualmente~\cite{forti2024}. Para além do volume, a periculosidade desses resíduos é um fator crítico. Um estudo aprofundado sobre sua composição detalha a presença de numerosas substâncias tóxicas, como chumbo, mercúrio e cádmio, que, quando descartadas incorretamente, representam uma séria ameaça à saúde humana e aos ecossistemas~\cite{perkins2014}.

Este cenário reflete um desafio sistêmico maior. Segundo a Organização para a Cooperação e Desenvolvimento Econômico \(OCDE\), os atuais padrões de extração e uso de materiais são insustentáveis, e o setor de eletrônicos é um dos principais contribuintes para o esgotamento de recursos naturais finitos~\cite{oecd2019}. Portanto, tratar o problema do e-waste é uma dupla necessidade: por um lado, mitigar a contaminação ambiental e, por outro, romper com um modelo econômico extrativista e desperdiçador. A solução mais eficaz, alinhada aos princípios da economia circular, é intervir antes que um equipamento se torne resíduo.

Neste contexto, a reutilização emerge como a estratégia prioritária, pois conserva a energia e os materiais embutidos na fabricação do produto, estendendo sua vida útil ao máximo. É a partir desta premissa que este trabalho se desenvolve, propondo uma solução tecnológica para um problema gerado pela própria tecnologia. O objetivo é criar uma ferramenta que não apenas ofereça um destino adequado para eletrônicos subutilizados, mas que também transforme um passivo ambiental em um ativo social, promovendo, como um valioso benefício secundário, a inclusão digital.

\section[MOTIVAÇÃO]{MOTIVAÇÃO}

A motivação central deste projeto nasce da constatação de uma grande falha logística e cultural no ciclo de vida dos produtos eletrônicos. Para um cidadão ou uma empresa que substitui um equipamento funcional, o processo de descarte sustentável é frequentemente opaco, inconveniente e desestruturado. Na ausência de canais claros e confiáveis para a doação ou o reaproveitamento, a opção mais fácil acaba sendo o descarte inadequado, que alimenta o crescente volume de e-waste.

Por outro lado, existe uma demanda latente por esses mesmos dispositivos. A tecnologia pode, ironicamente, resolver um problema que ela mesma ajudou a criar. A proposta é desenvolver a \textbf{Orbitar}, uma aplicação web que atue como uma ponte digital, conectando quem deseja descartar um eletrônico de forma consciente com quem pode dar a ele uma nova vida, tornando o processo de reutilização simples, seguro e geograficamente acessível.

\section[JUSTIFICATIVA]{JUSTIFICATIVA}

A relevância deste trabalho se justifica primariamente sob a ótica ambiental, pela necessidade urgente de se contrapor ao modelo de consumo linear. A importância de tratar o e-waste reside em seu duplo impacto negativo: primeiro, a contaminação ambiental e os riscos à saúde pública causados por seus componentes tóxicos, conforme detalhado por~\citeonline{perkins2014}; segundo, o esgotamento de recursos naturais finitos, desperdiçados no descarte prematuro de equipamentos. A plataforma \textbf{Orbitar} alinha-se diretamente aos princípios da economia circular~\cite{ellen2023}, promovendo a estratégia mais nobre e eficiente de gestão de resíduos: a reutilização. Ao estender a vida útil de um produto, evitam-se os impactos de todo o ciclo de produção de um novo e do descarte de um antigo.

Como um benefício social derivado, ao intervir no ciclo de descarte, a plataforma gera um impacto secundário de grande relevância, abordando a persistente desigualdade digital no Brasil. Dados da pesquisa TIC Domicílios~\cite{cgi2024} mostram que a posse de um computador ainda é um forte indicador de oportunidade socioeconômica. Portanto, a proposta se justifica por criar um mecanismo que ataca a raiz do problema do descarte inadequado e, como consequência, transforma um passivo ambiental em um ativo para a inclusão social.

\section[OBJETIVOS]{OBJETIVOS}

Diante do cenário exposto, os objetivos deste trabalho são definidos a seguir.

% --- CORREÇÃO DE ESTRUTURA: Títulos das subseções corrigidos para maior clareza ---
\subsection[OBJETIVO GERAL]{OBJETIVO GERAL}

Desenvolver a Orbitar, uma aplicação web focada em promover a economia circular e a sustentabilidade, que conecte doadores e receptores de produtos eletrônicos de forma organizada, segura e localizada, a fim de mitigar o descarte inadequado de e-waste e estender a vida útil da tecnologia.

\subsection[OBJETIVOS ESPECÍFICOS]{OBJETIVOS ESPECÍFICOS}

\begin{enumerate}[label={\alph*}]
    \item Realizar um levantamento conceitual sobre a gestão de e-waste e os princípios da economia circular, e analisar as tecnologias de desenvolvimento web a serem utilizadas (C\#\, .NET, Angular e SQL Server);
    \item Modelar a arquitetura da solução, incluindo o projeto do banco de dados relacional e a definição das interfaces de comunicação via API REST;
    \item Desenvolver o back-end da aplicação em C\# com a plataforma .NET, implementando as regras de negócio, o sistema de autenticação e o gerenciamento de dados;
    \item Desenvolver o front-end da aplicação com Angular, construindo uma interface de usuário responsiva, intuitiva e acessível;
    \item Implementar e validar o fluxo completo de doação, desde o cadastro e filtragem por localidade até a reserva temporária do item e a confirmação de entrega.
\end{enumerate}

\section[ORGANIZAÇÃO DO TRABALHO]{ORGANIZAÇÃO DO TRABALHO}

O restante deste trabalho está organizado da seguinte forma: o Capítulo 2 apresenta a fundamentação teórica, abordando os conceitos de e-waste, economia circular e as tecnologias que servem de base para o projeto. O Capítulo 3 descreve detalhadamente a metodologia empregada, incluindo a arquitetura do sistema, o modelo de dados e as ferramentas utilizadas no desenvolvimento. No Capítulo 4, são apresentados os resultados, com a demonstração das funcionalidades da plataforma. Finalmente, o Capítulo 5 traz as considerações finais, discutindo as conclusões do estudo, as limitações encontradas e as sugestões para trabalhos futuros.

\chapter[FUNDAMENTAÇÃO TEÓRICA]{FUNDAMENTAÇÃO TEÓRICA}

Este capítulo apresenta as bases conceituais e tecnológicas que sustentam o desenvolvimento da plataforma Orbitar. A exposição parte do fenômeno ambiental que motiva o projeto, avança para a discussão sobre arquiteturas de software e, por fim, detalha as tecnologias específicas empregadas na sua implementação.

\section[O DESAFIO DO LIXO ELETRÔNICO E A ECONOMIA CIRCULAR]{O DESAFIO DO LIXO ELETRÔNICO E A ECONOMIA CIRCULAR}

A proposta de valor do projeto reside na aplicação de princípios de sustentabilidade para resolver um problema gerado pelo próprio ciclo de consumo tecnológico. Resíduos de Equipamentos Elétricos e Eletrônicos \(REEE\), ou e-waste, compreendem todos os produtos eletrônicos ao final de sua vida útil. Este problema é definido tanto por sua escala massiva quanto por sua natureza perigosa.

Primeiramente, a escala do problema é evidenciada pelos dados do \textit{Global E-waste Monitor 2024}. O relatório indica que o mundo gerou 62 milhões de toneladas de lixo eletrônico em um único ano, um aumento de 82\% desde 2010, enquanto menos de um quarto \(22,3\%\) desse volume foi formalmente coletado e reciclado~\cite{forti2024}. Essa disparidade demonstra a ineficácia dos sistemas atuais de gestão de resíduos e a urgência de novas abordagens.

Em segundo lugar, a periculosidade do e-waste define sua gravidade. Conforme detalhado por~\citeonline{perkins2014}, os componentes eletrônicos contêm uma mistura complexa de materiais valiosos e substâncias perigosas. Elementos como chumbo em soldas, mercúrio em telas de LCD e cádmio em baterias recarregáveis são neurotoxinas e carcinógenos conhecidos. O descarte inadequado desses materiais em lixões ou sua queima a céu aberto liberam essas toxinas no ar, solo e água, resultando em contaminação ambiental duradoura e graves riscos à saúde das comunidades próximas.

A importância de tratar o problema do e-waste é reforçada pela estrutura da economia circular, que oferece um caminho viável para a sustentabilidade. Em oposição ao modelo linear de ``extrair-produzir-descartar'', a economia circular propõe um sistema restaurador e regenerativo por design. O objetivo é manter produtos, componentes e materiais em seu mais alto nível de utilidade e valor o tempo todo~\cite{ellen2023}. Dentro de seus princípios, a reutilização se destaca como uma das estratégias mais eficientes, pois demanda menos energia e recursos do que a reciclagem. A plataforma \textbf{Orbitar} foi concebida como uma ferramenta para operacionalizar esse princípio, criando um mercado secundário local e acessível que intercepta dispositivos antes que se tornem lixo, reinserindo-os no sistema produtivo e social.

\section[ARQUITETURA DE APLICAÇÕES WEB MODERNAS]{ARQUITETURA DE APLICAÇÕES WEB MODERNAS}

A solução foi concebida como uma aplicação web moderna, projetada para ser escalável, resiliente e acessível.

A arquitetura base é a cliente-servidor, um paradigma que separa as responsabilidades da aplicação. O cliente \(front-end\), executado no navegador do usuário, é responsável pela camada de apresentação e interação. O servidor (back-end) concentra a lógica de negócio, o acesso aos dados e as regras de segurança. Essa separação (desacoplamento) é fundamental, pois permite que as equipes de desenvolvimento do front-end e do back-end trabalhem de forma independente, além de viabilizar que diferentes tipos de clientes (como um futuro aplicativo móvel) consumam os mesmos serviços do servidor.

A comunicação entre cliente e servidor é mediada por uma Interface de Programação de Aplicações \(API\). Foi adotado o padrão arquitetural REST (Representational State Transfer), definido na dissertação seminal de Roy Fielding~\cite{fielding2000}. Uma API RESTful organiza a comunicação em torno de recursos (ex: doadores, doações, dispositivos), que são manipulados através dos verbos padrão do protocolo HTTP (GET, POST, PUT, DELETE). Uma característica chave do REST é ser stateless (sem estado), o que significa que cada requisição do cliente para o servidor deve conter toda a informação necessária para ser processada, garantindo escalabilidade e confiabilidade. Os dados são majoritariamente trafegados no formato JSON (JavaScript Object Notation) devido à sua leveza e fácil interpretação por linguagens de programação.

\section[TECNOLOGIAS ADOTADAS NO PROJETO]{TECNOLOGIAS ADOTADAS NO PROJETO}

A seleção tecnológica para a plataforma Orbitar foi orientada pela robustez, escalabilidade e pela forte integração do ecossistema de ferramentas escolhido.

Para o desenvolvimento do back-end, a escolha foi a linguagem C\# em conjunto com a plataforma .NET. Mantida pela Microsoft, a .NET é um framework de desenvolvimento de código aberto, multiplataforma e de alto desempenho, ideal para a construção de aplicações web e APIs robustas. Sua arquitetura moderna, forte tipagem e vasto conjunto de bibliotecas permitem a criação de sistemas seguros e escaláveis, enquanto o framework ASP.NET Core simplifica drasticamente a construção de APIs RESTful, permitindo que o desenvolvedor se concentre nas regras de negócio da aplicação.

A interface com o usuário \(front-end\) foi construída com Angular, um framework mantido pelo Google para o desenvolvimento de Single-Page Applications \(SPAs\). O Angular utiliza TypeScript, um superset do JavaScript que adiciona tipagem estática ao código, resultando em um desenvolvimento mais seguro e de fácil manutenção. Sua arquitetura baseada em componentes permite a criação de uma interface modular e reutilizável, o que é ideal para a complexidade e a evolução planejada para a plataforma.

Para a persistência dos dados, optou-se pelo Microsoft SQL Server, um Sistema de Gerenciamento de Banco de Dados \(SGBD\) relacional altamente robusto e confiável. Sua forte integração com o ecossistema .NET, por meio de ferramentas como o Entity Framework, otimiza o desenvolvimento da camada de acesso a dados. O SQL Server é reconhecido por sua performance e por garantir a integridade e consistência das informações através de transações ACID (Atomicidade, Consistência, Isolamento e Durabilidade), um requisito crucial para uma aplicação que gerencia o ciclo de vida de doações, onde cada etapa deve ser registrada de forma segura.

\begin{figure}[!h]
\centering
\caption{Escreva aqui o titulo da sua figura}
\includegraphics[scale=0.5]{UNIFESO.png} %coloque suas figuras na pasta imagem e mude o nome do aqui nessa parte. utilize o comando scale para alterar a escala da imagem
% AVISO: A compilação irá falhar se a chave de citação 'unifeso2025' não existir
% no seu arquivo bibliografia.bib.
\legend{Fonte:~\citeauthoronline{unifeso2025},~\citeyear{unifeso2025}}
\label{unifeso}% a escreva o nome qualquer a sua figura. Esse nome sera usado para referenciar a figura com o comando \ref{unifeso}
\end{figure}

Utilize o comando~\ref{unifeso} para referenciar as suas Figuras.
exemplo: Figura~\ref{unifeso}.

\chapter[METODOLOGIA E DESENVOLVIMENTO]{METODOLOGIA E DESENVOLVIMENTO}

Escreva aqui a metodologia adotada no seu trabalho

% ----------------------------------------------------------
% Considerações Finais
% ----------------------------------------------------------
\chapter{Conclusão}

Escreva a sua conclusão do seu trabalho


% ----------------------------------------------------------
% ELEMENTOS PÓS-TEXTUAIS
% ----------------------------------------------------------
\postextual{}
% ----------------------------------------------------------

% ----------------------------------------------------------
% Referências bibliográficas
% ----------------------------------------------------------
\bibliographystyle{abntex2-alf} % Define o estilo ABNT alfabético
\bibliography{bibliografia}  


% ----------------------------------------------------------
% Apêndices
% ----------------------------------------------------------

% ---
% Inicia os apêndices
% ---
\begin{apendicesenv}
	
% Imprime uma página indicando o início dos apêndices
\partapendices{}
	
\chapter{Exemplo}
% ----------------------------------------------------------
	
\end{apendicesenv}
% ---

% ----------------------------------------------------------
% Anexos
% ----------------------------------------------------------

% ---
% Inicia os anexos
% ---
\begin{anexosenv}
	
	% Imprime uma página indicando o início dos anexos
	\partanexos{}
	% ---
    \chapter{Exemplo}
    % ---
	
\end{anexosenv}

\end{document}