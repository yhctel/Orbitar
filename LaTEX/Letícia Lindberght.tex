\documentclass[
	% -- opções da classe memoir --
	12pt,				% tamanho da fonte
	openright,			% capítulos começam em pág ímpar (insere página vazia caso preciso)
	oneside,			% para impressão em frente e verso. Oposto a oneside
	a4paper,			% tamanho do papel.
	% -- opções da classe abntex2 --
	english,			% idioma adicional para hifenização
	brazil				% o último idioma é o principal do documento
	]{abntex2}


% --- PACOTES BÁSICOS ---
\usepackage{mathptmx}			% Usa a fonte Times New Roman
\usepackage[T1]{fontenc}		% Selecao de codigos de fonte.
\usepackage[utf8]{inputenc}		% Codificacao do documento (acentos)
\usepackage{lastpage}			% Usado pela Ficha catalográfrica
\usepackage{indentfirst}		% Indenta o primeiro parágrafo de cada seção.
\usepackage{graphicx}			% Inclusão de gráficos
\usepackage{subcaption}			% Inclusão de gráficos lado a lado
\usepackage{tabularx,ragged2e}	% Para inserir tabelas
\usepackage{multirow}			% Para mesclar células
\usepackage[dvipsnames,table,xcdraw]{xcolor}
\usepackage{fancyvrb}			% Permite adicionar arquivos de texto
\usepackage[portuguese, ruled, linesnumbered]{algorithm2e} % Uso de algoritmos
\usepackage{amsfonts}			% Permite usar notação de conjuntos
\usepackage{amsmath}			% Permite citar equações
\usepackage{amsthm}				% Permite criar teoremas e experimentos
\usepackage[font={bf, small}, labelsep=endash, labelfont=bf]{caption}
\usepackage{cancel}				% Permite fazer expressão tendendo a zero
\usepackage{epstopdf}			% Converte eps para pdf
\usepackage[final]{pdfpages}
\usepackage{hyphenat}
\usepackage{enumitem}
\usepackage{lipsum}
\usepackage{modelo-ufpa/ufpa}

% --- AJUSTE PARA NBR 10520:2023 ---
% Adicionada a opção 'abnt-cite-style=authoryear-lowcase' para que as citações
% (e.g., FULANO, 2023) apareçam como (Fulano, 2023), conforme a nova norma.
\usepackage[alf, abnt-emphasize=bf, abnt-show-options=doi=on,url=on, abnt-cite-style=authoryear-lowcase]{abntex2cite}



% --- CONFIGURAÇÕES DE LAYOUT E TÍTULOS ---

% Altera a fonte das seções (padrão abntex2) para não ser mais negrito
\renewcommand{\ABNTEXchapterfontsize}{\normalsize}
\renewcommand{\ABNTEXsectionfontsize}{\normalsize}
\renewcommand{\ABNTEXsubsectionfontsize}{\normalsize}
\renewcommand{\ABNTEXsectionfont}{}
\renewcommand{\ABNTEXsubsectionfont}{}

% Seções
\renewcommand{\cftsectionindent}{2.5em}
\renewcommand{\cftsectionnumwidth}{3.5em}

% Subseções
\renewcommand{\cftsubsectionindent}{5em}
\renewcommand{\cftsubsectionnumwidth}{4.5em}


% --- ESTILO DE PÁGINA ---
\makepagestyle{ultima}
\makeoddhead{ultima}{}{}{\thepage}
\makeevenhead{ultima}{}{}{\thepage}
\makeheadrule{ultima}{0pt}{0pt}
\makefootrule{ultima}{0pt}{0pt}

% --- CAMINHO DAS IMAGENS ---
\graphicspath{{imagens/}}

% --- AMBIENTE DE EXPERIMENTOS ---
\theoremstyle{definition}
\newtheorem{experimento}{Experimento}[section]
\newcommand{\experimentoautorefname}{Experimento}

% --- INFORMAÇÕES DA CAPA E FOLHA DE ROSTO (COMANDOS DO PACOTE UFPA) ---
\universidade{Centro Universitário Serra dos Órgãos \- UNIFESO}
\instituto{Direção Acadêmica de Ciências Humanas e Tecnológicas \- DACHT}
\curso{Curso de Bacharelado em Ciência da Computação}
\titulo{Orbitar: Uma plataforma web para o reaproveitamento de eletrônicos e promoção da economia circular}
\autor{Letícia Lindberght da Costa}
\local{Teresópolis}
\data{2025}
\orientador{Prof. Nome do Orientador}
\tipotrabalho{Monografia}
\preambulo{Trabalho de Conclusão de Curso apresentado ao Centro Universitário Serra dos Órgãos como requisito obrigatório para obtenção do título de Bacharel em Ciência da Computação.}
\sobrenome{Lindberght}
\nome{Letícia}
\palavraschave{Palavra-chave1, Palavra-chave2, Palavra-chave3}

\datadadefesa{Data da Defesa: 28 de Novembro de 2025}
\faculdadedoorientador{Faculdade do Orientador}
\titulacaodoorientador{MSc}
\faculdadedocoorientador{Faculdade do Coorientador}
\titulacaodocoorientador{DSc}
\primeiromembrodabanca{Nome do Primeiro Membro da Banca}
\titulacaodoprimeiromembro{DSc}
\faculdadedoprimeiromembrodabanca{Faculdade do Primeiro Membro da Banca}
\segundomembrodabanca{Nome do Segundo Membro da Banca}
\titulacaodosegundomembro{DSc}
\faculdadedosegundomembrodabanca{Faculdade do Segundo Membro da Banca}

% --- CONFIGURAÇÕES DE APARÊNCIA DO PDF ---
\definecolor{blue}{RGB}{41,5,195}
\hypersetup{
    pdftitle={\texorpdfstring{\imprimirtitulo}{Orbitar \- Plataforma Web de Reaproveitamento de Eletrônicos}}, 
    pdfauthor={\texorpdfstring{\imprimirautor}{Letícia Lindberght da Costa}},
    pdfsubject={\texorpdfstring{\imprimirpreambulo}{Trabalho de Conclusão de Curso}},
    pdfkeywords={\texorpdfstring{\imprimirpalavraschave}{Palavra-chave1, Palavra-chave2, Palavra-chave3}},
    colorlinks=true,
    linkcolor=black,
    citecolor=black,
    filecolor=magenta,
    urlcolor=black,
    bookmarksdepth=4,
    breaklinks=true
}

% --- ESPAÇAMENTOS ---
\setlength{\parindent}{1.5cm}
\setlength{\parskip}{0.2cm}

% --- COMPILA O ÍNDICE REMISSIVO ---
\makeindex

% ----
% Início do documento
% ----
\begin{document}

% Seleciona o idioma
\selectlanguage{brazil}

% Retira espaço extra obsoleto entre as frases.
\frenchspacing 

% ----------------------------------------------------------
% ELEMENTOS PRÉ-TEXTUAIS
% ----------------------------------------------------------
\pretextual{}

% --- Capa ---
\imprimircapa{}

% --- Folha de rosto ---
\imprimirfolhaderosto{}

% --- Ficha Catalográfica ---
% \begin{fichacatalografica}
%    \includepdf{fichacatalografica.pdf}
% \end{fichacatalografica}
\newpage

% --- Folha de Aprovação ---
\begin{folhadeaprovacao}
\imprimirfolhadeaprovacao{}
\end{folhadeaprovacao}

% --- Dedicatória ---
\begin{dedicatoria}
    \vspace*{\fill}
    \flushright{}
    \textit{Este trabalho é dedicado às crianças adultas que,
	quando pequenas, sonharam em se tornar cientistas.
	\- Lauro César em abnTeX2}
\end{dedicatoria}

% --- Agradecimentos ---
\begin{agradecimentos}
Escreva aqui os seus agradecimentos da maneira que melhor lhe convêm.
\end{agradecimentos}

% --- Epígrafe ---
\begin{epigrafe}
    \vspace*{\fill}
	\begin{flushright}
		\textit{``Que todos os nossos esforços estejam sempre focados no desafio à impossibilidade.\\ Todas as grandes conquistas humanas vieram daquilo que parecia impossível''.\\
		(Charles Chaplin)}
	\end{flushright}
\end{epigrafe}

% --- Resumos ---
\setlength{\absparsep}{18pt}
\begin{resumo}
Escreva aqui o resumo do seu trabalho.

\textbf{Palavras-chave}: \imprimirpalavraschave{}
\end{resumo}

\begin{resumo}[Abstract]
 \begin{otherlanguage*}{english}
   Escreva aqui o seu abstract (Resumo em inglês)
   \vspace{\onelineskip}
   \noindent 
   \textbf{Keywords}: Keywords1. Keywords2. Keywords3.
 \end{otherlanguage*}
\end{resumo}

% --- Listas ---
\pdfbookmark[0]{\listfigurename}{lof}
\listoffigures*
\cleardoublepage{}

\pdfbookmark[0]{\listofquadrosname}{loq}
\listofquadros*
\cleardoublepage{}

\pdfbookmark[0]{\listtablename}{lot}
\listoftables*
\cleardoublepage{}

\pdfbookmark[0]{\listalgorithmcfname}{loa}
\imprimirlistadealgoritmos{}
\cleardoublepage{}

\begin{siglas}
  \item[ABNT] Associação Brasileira de Normas Técnicas
  \item[TIC] Tecnologias da Informação e Comunicação
\end{siglas}

% --- Sumário ---
\pdfbookmark[0]{\contentsname}{toc}
\tableofcontents*
\cleardoublepage{}

% ----------------------------------------------------------
% ELEMENTOS TEXTUAIS
% ----------------------------------------------------------
\textual{}
\pagestyle{simple}

% Capítulos
\chapter[Introdução]{Introdução}

As Tecnologias da Informação e Comunicação (TIC) são pilares da sociedade contemporânea, catalisando o progresso social e econômico. Contudo, o modelo de produção que as sustenta, predominantemente linear e focado no consumo cíclico, gerou um passivo ambiental de proporções críticas: o lixo eletrônico, ou e-waste. Este fenômeno é o resultado direto de um sistema caracterizado pela obsolescência programada e por padrões de consumo insustentáveis, onde equipamentos são descartados em um ritmo alarmante, muitas vezes ainda em plenas condições de uso.

A magnitude deste problema é documentada por múltiplas fontes globais. O relatório \textit{Global E-waste Monitor 2024} revela que a geração de lixo eletrônico atingiu um recorde histórico, crescendo a uma taxa cinco vezes superior à da reciclagem documentada, com dezenas de milhões de toneladas descartadas anualmente~\cite{forti2024}. Para além do volume, a periculosidade desses resíduos é um fator crítico. Um estudo aprofundado sobre sua composição detalha a presença de numerosas substâncias tóxicas, como chumbo, mercúrio e cádmio, que, quando descartadas incorretamente, representam uma séria ameaça à saúde humana e aos ecossistemas~\cite{perkins2014}.

Este cenário reflete um desafio sistêmico maior. Segundo a Organização para a Cooperação e Desenvolvimento Econômico (OCDE), os atuais padrões de extração e uso de materiais são insustentáveis, e o setor de eletrônicos é um dos principais contribuintes para o esgotamento de recursos naturais finitos~\cite{oecd2019}. Portanto, tratar o problema do e-waste é uma dupla necessidade: por um lado, mitigar a contaminação ambiental e, por outro, romper com um modelo econômico extrativista e desperdiçador. A solução mais eficaz, alinhada aos princípios da economia circular, é intervir antes que um equipamento se torne resíduo.

Neste contexto, a reutilização emerge como a estratégia prioritária, pois conserva a energia e os materiais embutidos na fabricação do produto, estendendo sua vida útil ao máximo. É a partir desta premissa que este trabalho se desenvolve, propondo uma solução tecnológica para um problema gerado pela própria tecnologia. O objetivo é criar uma ferramenta que não apenas ofereça um destino adequado para eletrônicos subutilizados, mas que também transforme um passivo ambiental em um ativo social, promovendo, como um valioso benefício secundário, a inclusão digital.

\section[Motivação]{Motivação}

A motivação central deste projeto nasce da constatação de uma grande falha logística e cultural no ciclo de vida dos produtos eletrônicos. Para um cidadão ou uma empresa que substitui um equipamento funcional, o processo de descarte sustentável é frequentemente opaco, inconveniente e desestruturado. Estudos sobre a gestão de resíduos no Brasil apontam que a falta de pontos de coleta acessíveis e a desinformação da população são barreiras significativas para a destinação correta dos equipamentos~\cite{GreenEletron2022}. Na ausência de canais claros e confiáveis para a doação ou o reaproveitamento, a opção mais fácil acaba sendo o descarte inadequado, que alimenta o crescente volume de e-waste~\cite{forti2024}.

Por outro lado, existe uma demanda latente e socialmente urgente por esses mesmos dispositivos. A exclusão digital no Brasil é um reflexo direto da desigualdade socioeconômica, onde o alto custo de novos equipamentos representa um grande obstáculo para milhões de famílias de baixa renda. A própria pesquisa TIC Domicílios, ao mapear a posse de dispositivos por faixa de renda, evidencia essa barreira econômica, mostrando que, enquanto o acesso à internet é quase universal nas classes mais altas, a posse de um computador — ferramenta essencial para a educação e qualificação profissional — ainda é significativamente menor nas classes D e E~\cite{cgi2024}. Essa realidade cria um paradoxo: de um lado, dispositivos funcionais são descartados prematuramente por falta de canais; de outro, uma vasta parcela da população carece de acesso a essa mesma tecnologia por não ter poder de compra.

Nesse contexto, a tecnologia pode, ironicamente, resolver um problema que ela mesma ajudou a criar. A proposta é desenvolver a \textbf{Orbitar}, uma aplicação web que atue como uma ponte digital, conectando quem deseja descartar um eletrônico de forma consciente com quem pode dar a ele uma nova vida, tornando o processo de reutilização simples, seguro e geograficamente acessível.

\section[Justificativa]{Justificativa}

A relevância deste trabalho se justifica primariamente sob a ótica ambiental, pela necessidade urgente de se contrapor ao modelo de consumo linear. A importância de tratar o e-waste reside em seu duplo impacto negativo: primeiro, a contaminação ambiental e os riscos à saúde pública causados por seus componentes tóxicos, conforme detalhado por~\cite{perkins2014}; segundo, o esgotamento de recursos naturais finitos, desperdiçados no descarte prematuro de equipamentos. A plataforma \textbf{Orbitar} alinha-se diretamente aos princípios da economia circular~\cite{ellen2023}, promovendo a estratégia mais nobre e eficiente de gestão de resíduos: a reutilização. Ao estender a vida útil de um produto, evitam-se os impactos de todo o ciclo de produção de um novo e do descarte de um antigo.

Como um benefício social derivado, ao intervir no ciclo de descarte, a plataforma gera um impacto secundário de grande relevância, abordando a persistente desigualdade digital no Brasil. Dados da pesquisa TIC Domicílios~\cite{cgi2024} mostram que a posse de um computador ainda é um forte indicador de oportunidade socioeconômica. Portanto, a proposta se justifica por criar um mecanismo que ataca a raiz do problema do descarte inadequado e, como consequência, transforma um passivo ambiental em um ativo para a inclusão social.

\section[Objetivos]{Objetivos}

Diante do cenário exposto, os objetivos deste trabalho são definidos a seguir.

\subsection[Objetivo Geral]{Objetivo Geral}

Desenvolver a Orbitar, uma aplicação web focada em promover a economia circular e a sustentabilidade, que conecte doadores e receptores de produtos eletrônicos de forma organizada, segura e localizada, a fim de mitigar o descarte inadequado de e-waste e estender a vida útil da tecnologia.

\subsection[Objetivos Específicos]{Objetivos Específicos}

\begin{enumerate}[label={\alph*}]
    \item Realizar um levantamento conceitual sobre a gestão de e-waste e os princípios da economia circular, e analisar as tecnologias de desenvolvimento web a serem utilizadas (C\#, .NET, Angular e SQL Server);
    \item Modelar a arquitetura da solução, incluindo o projeto do banco de dados relacional e a definição das interfaces de comunicação via API REST;
    \item Desenvolver o back-end da aplicação em C\# com a plataforma .NET, implementando as regras de negócio, o sistema de autenticação e o gerenciamento de dados;
    \item Desenvolver o front-end da aplicação com Angular, construindo uma interface de usuário responsiva, intuitiva e acessível;
    \item Implementar e validar o fluxo completo de doação, desde o cadastro e filtragem por localidade até a reserva temporária do item e a confirmação de entrega.
\end{enumerate}

\section[Organização do Trabalho]{Organização do Trabalho}

O restante deste trabalho está organizado da seguinte forma: o Capítulo 2 apresenta a fundamentação teórica, abordando os conceitos de e-waste, economia circular e as tecnologias que servem de base para o projeto. O Capítulo 3 descreve detalhadamente a metodologia empregada, incluindo a arquitetura do sistema, o modelo de dados e as ferramentas utilizadas no desenvolvimento. No Capítulo 4, são apresentados os resultados, com a demonstração das funcionalidades da plataforma. Finalmente, o Capítulo 5 traz as considerações finais, discutindo as conclusões do estudo, as limitações encontradas e as sugestões para trabalhos futuros.

\chapter{FUNDAMENTAÇÃO TEÓRICA}

Este capítulo apresenta os conceitos essenciais que sustentam este trabalho. Inicia-se com a contextualização do lixo eletrônico como um desafio global, detalhando sua composição e seus impactos. Em seguida, introduz-se a economia circular como o paradigma de solução, com foco na estratégia de reutilização. A discussão então conecta a reutilização de eletrônicos à promoção da inclusão digital. Finalmente, são abordadas as tecnologias de desenvolvimento web selecionadas para a construção da plataforma Orbitar.

\section{O Problema do Lixo Eletrônico (E-waste)}

A sociedade moderna, impulsionada pelo avanço tecnológico e pelo consumo acelerado, enfrenta um dos maiores desafios ambientais e sociais: o lixo eletrônico. Caracterizado como Resíduos de Equipamentos Elétricos e Eletrônicos (REEE), o e-waste engloba uma vasta gama de produtos que, ao fim de sua vida útil, são descartados de maneira inadequada, gerando severas consequências.

\subsection{Definição e Escopo}

O lixo eletrônico, ou e-waste, refere-se a qualquer equipamento elétrico ou eletrônico que se tornou obsoleto, quebrado ou indesejado e que, por isso, é descartado. A sua definição, conforme consolidada por órgãos como as Nações Unidas e a legislação europeia, abrange desde itens de uso doméstico até equipamentos industriais, refletindo a ubiquidade da tecnologia na vida contemporânea~\cite{Forti2020, EUWEEE2012}.

As categorias de REEE são amplas, indo muito além dos comumente associados celulares e computadores. A Diretiva WEEE (Waste from Electrical and Electronic Equipment) da União Europeia, por exemplo, classifica o e-waste em grupos distintos, tais como:
\begin{itemize}
    \item \textbf{Grandes Eletrodomésticos:} geladeiras, máquinas de lavar, fogões;
    \item \textbf{Pequenos Eletrodomésticos:} torradeiras, aspiradores de pó, cafeteiras;
    \item \textbf{Equipamentos de Informática e Telecomunicações:} computadores, notebooks, tablets, smartphones, impressoras;
    \item \textbf{Equipamentos de Consumo:} televisores, rádios, câmeras digitais;
    \item \textbf{Ferramentas Elétricas e Eletrônicas:} furadeiras, serras elétricas;
    \item \textbf{Brinquedos, Equipamentos Esportivos e de Lazer:} consoles de videogame, jogos eletrônicos~\cite{EUWEEE2012}.
\end{itemize}
Essa diversidade demonstra a amplitude do problema e a necessidade de soluções abrangentes para o gerenciamento de um fluxo de resíduos tão variado.

\subsection{Composição e Periculosidade}

A complexidade do lixo eletrônico reside em sua composição heterogênea, que inclui tanto materiais valiosos quanto substâncias altamente tóxicas. Dispositivos eletrônicos são verdadeiros ``minérios urbanos'', contendo em sua estrutura metais preciosos como ouro, prata, platina e cobre, que são valiosos para a indústria~\cite{Robinson2009}.

No entanto, a grande preocupação reside na presença de substâncias perigosas. Equipamentos eletrônicos frequentemente contêm metais pesados como chumbo, mercúrio e cádmio, além de retardantes de chama bromados (BFRs)~\cite{Widmer2005}. Estudos detalham como a exposição a esses componentes, especialmente o chumbo e o mercúrio, pode causar danos neurológicos, renais e ao desenvolvimento, tornando o descarte inadequado não apenas um problema ambiental, mas uma grave questão de saúde pública~\cite{perkins2014b}.

\subsection{Impactos Ambientais e na Saúde Humana}

O descarte inadequado do lixo eletrônico desencadeia uma cascata de impactos negativos. Do ponto de vista ambiental, quando esses resíduos são depositados em aterros, suas substâncias tóxicas infiltram-se no ambiente, contaminando o solo e os lençóis freáticos, um risco global documentado por avaliações abrangentes sobre o tema~\cite{Robinson2009}. Do ponto de vista da saúde pública, a situação é igualmente crítica: a prática informal de queima de e-waste para a extração de metais valiosos libera fumaças tóxicas que causam poluição atmosférica severa e expõem diretamente as populações que manipulam esses materiais a doenças respiratórias e intoxicações~\cite{Heacock2016}.

\subsection{Causas do Aumento Exponencial}

O crescimento exponencial do volume de e-waste é um reflexo direto de um modelo de produção e consumo insustentável. A principal causa reside no modelo de economia linear (“extrair, produzir, usar e descartar”). Associada a isso, a obsolescência programada — estratégia de projetar produtos com vida útil artificialmente curta — e a cultura do consumo cíclico impulsionam a substituição acelerada de dispositivos. Estudos sobre o cenário brasileiro apontam que esses fatores, combinados com a expansão do acesso a novas tecnologias, são os principais vetores para o aumento massivo do descarte prematuro de equipamentos~\cite{Pereira2018, Forti2020}.

\section{A Economia Circular como Solução Estratégica}

Diante do cenário de esgotamento de recursos e da crescente geração de resíduos, a economia circular emerge como um paradigma transformador, propondo uma alternativa sustentável ao modelo econômico linear.

\subsection{O Contraste com a Economia Linear}

A economia linear baseia-se no modelo de ``extrair, produzir, usar e descartar''. Este modelo é intrinsecamente insustentável, pois depende de recursos finitos e gera resíduos em todas as etapas, resultando em poluição e perda de valor material.

\subsection{Princípios da Economia Circular}

Em contrapartida, a economia circular é um modelo regenerativo por design. Conforme definido pela Fundação Ellen MacArthur, ela se baseia em três princípios essenciais:

\begin{enumerate}
    \item Eliminar resíduos e poluição desde o princípio;
    \item Manter produtos e materiais em uso pelo maior tempo possível;
    \item Regenerar sistemas naturais~\cite{EMF2013}.
\end{enumerate}

Esses princípios visam desacoplar o crescimento econômico do consumo de recursos finitos.

\subsection{A Hierarquia dos 5 Rs da Sustentabilidade}

Para operacionalizar a economia circular, a hierarquia dos 5 Rs (Repensar, Recusar, Reduzir, Reutilizar e Reciclar) oferece um guia prático. Dentro dessa lógica, a \textbf{reutilização} é uma estratégia superior à reciclagem. Enquanto a reciclagem demanda energia para transformar o resíduo em matéria-prima, a reutilização conserva integralmente o valor agregado ao produto (energia, mão de obra, componentes), representando uma economia de recursos significativamente maior. Esta hierarquia é reconhecida tanto na literatura acadêmica quanto na legislação brasileira, por meio da Política Nacional de Resíduos Sólidos~\cite{Kirchherr2017, Brasil2010}. A plataforma Orbitar, ao focar na reutilização, alinha-se diretamente com este princípio central.

\section{Reutilização de Eletrônicos e o Combate à Exclusão Digital}

A estratégia de reutilização de equipamentos eletrônicos transcende os benefícios ambientais, atuando como um poderoso vetor de inclusão social no combate à exclusão digital.

\subsection{A Reutilização como Estratégia Prioritária}

Ao estender a vida útil de um dispositivo, evita-se a emissão de gases de efeito estufa e o consumo de recursos naturais necessários para a fabricação de um novo produto. A reutilização impede que um produto funcional se torne um resíduo, minimizando a pressão sobre aterros sanitários e o risco de contaminação~\cite{Kirchherr2017}.

\subsection{O Cenário da Exclusão Digital no Brasil}

O Brasil ainda enfrenta um cenário desafiador de exclusão digital. Dados da pesquisa TIC Domicílios, realizada pelo Comitê Gestor da Internet no Brasil (CGI.br), revelam que, embora o acesso à internet tenha se expandido, a posse de dispositivos como computadores ainda é desigual, especialmente entre as classes sociais mais baixas e em áreas rurais~\cite{cgi2024}. A falta de um computador em casa representa uma barreira substancial para a educação, a qualificação profissional e o pleno exercício da cidadania.

\subsection{A Tecnologia como Ponte para a Inclusão}

É nesse contexto que a reutilização de eletrônicos se torna um elo vital. A plataforma Orbitar surge como um mecanismo para transformar o que seria um passivo ambiental em um ativo de inclusão social. Dispositivos que seriam descartados ganham uma nova vida nas mãos de quem realmente precisa, funcionando como uma ponte tecnológica que democratiza o acesso e fortalece a educação e a empregabilidade, alinhando-se a modelos de negócio sustentáveis que geram valor social e ambiental~\cite{Geissdoerfer2017}.

\section{Tecnologias para o Desenvolvimento da Solução}


A construção da plataforma Orbitar exige uma arquitetura robusta, utilizando tecnologias modernas que garantam desempenho, segurança e manutenibilidade.

\subsection{Arquitetura de Aplicações Web Modernas}

As aplicações web contemporâneas são predominantemente construídas sob o modelo de arquitetura cliente-servidor. O \textbf{front-end} (cliente) é responsável pela interface com o usuário, enquanto o \textbf{back-end} (servidor) contém a lógica de negócio e o acesso aos dados~\cite{Pressman2014}. A comunicação entre essas camadas é realizada através de APIs (Application Programming Interfaces), sendo o padrão \textbf{REST (Representational State Transfer)} o mais comum por sua simplicidade e escalabilidade, permitindo o desacoplamento entre as partes~\cite{fielding2000}.

\subsection{Desenvolvimento Back-end com C\# e .NET}

Para o back-end, optou-se pela linguagem \textbf{C\#} e pela plataforma \textbf{.NET}. Esta escolha é justificada pela robustez, segurança e alto desempenho do ecossistema, mantido pela Microsoft. O framework ASP.NET Core, parte do .NET, é otimizado para a criação de APIs RESTful de alta performance, sendo multiplataforma e contando com vasto suporte da comunidade e de ferramentas que aceleram o desenvolvimento, como o Entity Framework Core para acesso a dados~\cite{dotnet2024}.

\subsection{Desenvolvimento Front-end com Angular}

Para o front-end, foi selecionado o framework \textbf{Angular}. Sua arquitetura baseada em componentes promove a reutilização de código e a manutenibilidade. Além disso, o Angular utiliza \textbf{TypeScript}, um superconjunto do JavaScript que adiciona tipagem estática, tornando o código mais seguro e menos propenso a erros em aplicações complexas. O ecossistema completo do Angular, mantido pelo Google, oferece ferramentas que otimizam todo o ciclo de desenvolvimento~\cite{angular2024}.

\subsection{Banco de Dados com SQL Server}

A persistência dos dados será realizada com o \textbf{Microsoft SQL Server}, um Sistema de Gerenciamento de Banco de Dados (SGBD) relacional. A escolha de um banco de dados relacional é estratégica devido à natureza estruturada dos dados da aplicação (usuários, produtos, doações), que se beneficia da integridade e consistência garantidas pelo modelo relacional. O SQL Server oferece garantias transacionais (ACID), segurança robusta e excelente integração com o ecossistema .NET, sendo ideal para a aplicação~\cite{sqlserver2024}.

\begin{figure}[!h]
\centering
\caption{Escreva aqui o titulo da sua figura}
\includegraphics[scale=0.5]{UNIFESO.png}
\legend{Fonte:~\citeauthoronline{unifeso2025},~\citeyear{unifeso2025}.} % Ponto final adicionado
\label{unifeso}
\end{figure}

Utilize o comando~\ref{unifeso} para referenciar as suas Figuras.
exemplo: Figura~\ref{unifeso}.

\chapter[METODOLOGIA E DESENVOLVIMENTO]{METODOLOGIA E DESENVOLVIMENTO}

Escreva aqui a metodologia adotada no seu trabalho.

\chapter{Conclusão}

Escreva a sua conclusão do seu trabalho.


% ----------------------------------------------------------
% ELEMENTOS PÓS-TEXTUAIS
% ----------------------------------------------------------
\postextual{}

% ----------------------------------------------------------
% Referências bibliográficas
% ----------------------------------------------------------
\nocite{angular2024, dotnet2024, sqlserver2024}
\bibliographystyle{abntex2-alf}
\bibliography{bibliografia}  

% ----------------------------------------------------------
% Apêndices
% ----------------------------------------------------------
\begin{apendicesenv}
\partapendices{}
\chapter{Exemplo de Apêndice}
\end{apendicesenv}

% ----------------------------------------------------------
% Anexos
% ----------------------------------------------------------
\begin{anexosenv}
\partanexos{}
\chapter{Exemplo de Anexo}
\end{anexosenv}

\end{document}